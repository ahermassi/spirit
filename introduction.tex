% (C) 2016 Jean Nassar. Some Rights Reserved
% Except where otherwise noted, this work is licensed under the Creative Commons Attribution-ShareAlike License, version 4
% Introduction
\chapter{Introduction}
\label{ch:intro}
This report is divided into five chapters.
In Chapter \ref{ch:intro}, the background, purpose of study, previous research, and the proposed method are presented.
In Chapter \ref{ch:spirit}, the new method is explained.
In Chapter \ref{ch:experiments}, the experiments are described.
Their results are discussed in Chapter \ref{ch:results}.
Finally, conclusions and proposals for future work are presented in \ref{ch:conclusion}.

\section{Unmanned Aerial Vehicles}
\Glspl{uav}, also known as drones, have been used since the mid-19th century for both military and private purposes.
Recently, multirotor \glspl{uav} have become common, and are now used recreationally and professionally.
Some of the applications of \glspl{uav} include:

\begin{itemize}
  \item capturing photos and videos
  \item racing
  \item site inspection
  \item agriculture and land assessment
  \item news gathering
  \item police surveillance
  \item exploration
  \item search and rescue, and disaster relief
\end{itemize}

Many \glspl{uav} can be programmed, but they are also often flown manually.

A modern \gls{uav}'s payload typically consists of at least one camera, an \gls{imu} with at least six \glspl{dof}, and an autopilot chip to aid with stability.
More sophisiticated \glspl{uav} might have additional cameras, or sensors for altitude and distance, including sensors for \gls{not:3d} reconstruction of the environment.
However, these features tend to require a larger, heavier \gls{uav}, which might cost more and be less maneuverable.

Teleoperated \glspl{uav} usually use either line of sight, where the pilot has to be able to see the drone directly, or \gls{fpv}, where the pilot uses a video downlink from the drone to control it.
The downlink can be displayed in video goggles or on a standard display.
Some setups allow the camera on the drone to gimbal, which can provide an immersive experience when combined with head tracking.

\gls{fpv}, however, is normally \gls{not:2d}, and does not afford the user any depth perception.
As a result, loss of situational (\emph{i.e.}, spatial) awareness can occur, and the boundaries of the vehicle with respect to the environment might not be known.
This is known to have contributed to multiple crashes into the environment in the past, with powerlines damaged,\cite{latimes2015} and people gravely injured.\cite{seattletimes2015,bbcnews2015}
Incidents of \glspl{uav} flying dangerously close to airports have also been reported.\cite{ctvnews2014}

Moreover, \gls{dos} can occur when flying far, or in a confined environment, causing dropped frames and choppy video.
This, in turn, can cause the operator to miss important cues and react too late, also causing an accident.

\section{Research objective}
The objective of this research is to increase the situational awareness of a \gls{uav} pilot using a single onboard monocular camera.
The system should be capable of functioning in a low bandwidth situation, and provide increased position and velocity control compared to \gls{fpv}.
Ideally, the system would also have a fast uptake speed, and would be independent of the controller used.

\section{Previous research}
Sugimoto \emph{et al.} introduced past-image view technology for mobile robots.\cite{sugimoto2005}
This technology enabled the operator to see the position and orientation of the robot with respect to its environment from a third-person perspective.

Images taken by a raised camera were saved in a buffer along with their position and time.
An optimal image was then found using an evaluation function considering several constraints, including \gls{fov} and the relative positions of the camera and robot.

If the bandwidth is low, old images can be reused, updated by an array of position data sent at a higher speed than the video.

Ito \emph{et al.} decreased choppiness by proposing wider \glspl{fov}, and zooming into the background and model as the robot went further into the image plane.\cite{ito2008} They also experimented with adding trajectory forecasting with unmanned ground vehicles.

Murata \emph{et al.} extended the concept to \gls{not:3d} when they applied it to mobile manipulators.\cite{murata2014}
They showed a significant decrease in position error in the vertical direction with a similar error in the longitudinal direction.

